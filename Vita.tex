% ------------------------------------------------------------------------------
% Use the ‘scrartcl’ class from KOMA-Script with:
% - 11pt font;
% - A4 paper with 20 cm × 28 cm type area;
% - Numbered entries in the TOC for all bibliographies;
% - No paragraph indentation or inter-paragraph spacing;
% - Customized spacing around ‘\section’ and ‘\subsection’;
% - Horizontal rule after ‘\section’; and
% - Empty page style.

\documentclass[%
version=last,%
fontsize=11pt,%
paper=A4,%
areasetadvanced,%
headinclude=false,%
footinclude=false,%
headlines=0,%
footlines=0,%
toc=bibnumbered,%
]{scrartcl}

\directlua{pdf.setminorversion(7)}% chktex 36
\areaset{20cm}{28cm}
\setparsizes{0pt}{0pt}{0pt plus 1fil}
\pagestyle{empty}

\RedeclareSectionCommand[%
afterindent=false,%
beforeskip=\baselineskip,%
afterskip=.75\baselineskip,%
]{section}
\RedeclareSectionCommand[%
afterindent=false,%
beforeskip=.75\baselineskip,%
afterskip=.5\baselineskip,%
]{subsection}

\makeatletter
\renewcommand{\sectionlinesformat}[4]{%
  \Ifstr{#1}{section}{%
    \parbox[t]{\linewidth}{%
      \raggedsection\@hangfrom{\hskip #2#3}{#4}\par%
      \kern-.75\ht\strutbox\rule{\linewidth}{.75pt}%
    }%
  }{%
    \@hangfrom{\hskip #2#3}{#4}%
  }%
}
\makeatother

% ------------------------------------------------------------------------------
% Load ‘xpatch’ so that it can be used to patch the ‘thesis’ driver from
% ‘biblatex’.

\RequirePackage{xpatch}

% ------------------------------------------------------------------------------
% Load ‘minted’ for highlighting of source code.

\RequirePackage[newfloat]{minted}

% ------------------------------------------------------------------------------
% Configure the document’s language and fonts.
%
% Typeset the document in en-US, but use DE for phone numbers; furthermore, load
% the ‘microtype’ package.

\RequirePackage[american]{babel}
\RequirePackage[style=american,autostyle,autopunct,strict]{csquotes}
\RequirePackage[country=DE]{phonenumbers}
\RequirePackage[shortcuts]{extdash}

\RequirePackage[final,babel]{microtype}

\RequirePackage{fontawesome5}

\babelfont{rm}[Language=Default]{Libertinus Serif}
\babelfont{sf}[Language=Default]{Libertinus Sans}
\babelfont{tt}[Language=Default,Scale=MatchLowercase]{Fantasque Sans Mono}

\newfontfamily\SpacedLSFamily[Language=Default,LetterSpace=4.0]{Libertinus Sans}
\NewDocumentCommand{\NameFont}{}{\bfseries\scshape\SpacedLSFamily}
\DeclareTextFontCommand{\PrintName}{\NameFont}

\DeclareTextFontCommand{\PrintISODate}{\ttfamily}
\DeclareTextFontCommand{\PrintInstitution}{\sffamily}
\DeclareTextFontCommand{\PrintDegree}{\bfseries\sffamily}
\DeclareTextFontCommand{\PrintPosition}{\bfseries\sffamily}

\newfontfamily\SCPFamily[Language=Default,Scale=MatchLowercase]{Source Code Pro}
\DeclareTextFontCommand{\TextSCP}{\SCPFamily}

\newfontfamily\DSFamily[Language=Default,Scale=MatchLowercase]{Dancing Script}
\DeclareTextFontCommand{\TextDS}{\DSFamily}

% ------------------------------------------------------------------------------
% Load ‘mathtools’ and ‘amsthm’ along with some additional packages for
% mathematical typesetting.

\RequirePackage{mathtools}
\RequirePackage{amsthm}
\RequirePackage[ruled,longend]{algorithm2e}
\RequirePackage{prftree}
\RequirePackage[warnings-off={mathtools-colon,mathtools-overbracket}]{unicode-math}
\mathtoolsset{mathic}
\unimathsetup{math-style=ISO,bold-style=ISO}
\setmathfont[Language=Default]{Libertinus Math}

% ------------------------------------------------------------------------------
% Colors.
\RequirePackage{xcolor}
\colorlet{framered}{red!75!white}
\colorlet{backred}{red!5!white}

% ------------------------------------------------------------------------------
% Enable extended support for graphics and load TikZ and some libraries.

\RequirePackage{graphicx}
\RequirePackage{tikz}
\usetikzlibrary{%
  arrows.meta,%
  calc,%
  fit,%
  patterns,%
  positioning,%
}
\tikzset{%
  surround/.style 2 args={%
    draw,%
    rectangle,%
    thick,%
    dotted,%
    fit=#1,%
    inner sep=8pt,
    label={[align=center,scale=0.75]#2},%
  },%
}

% ------------------------------------------------------------------------------
% Enhancements for arrays and tabulars.

\RequirePackage{array}
\RequirePackage{booktabs}
\RequirePackage{tabularx}

\newcolumntype{P}[1]{%
  >{%
    \hsize=\dimexpr#1\hsize\relax%
    \linewidth=\dimexpr\hsize\relax%
    \arraybackslash%
  }X%
}
\newcolumntype{M}[2]{%
  >{%
    \hsize=\dimexpr#1\hsize+#2\tabcolsep\relax%
    \linewidth=\dimexpr\hsize\relax%
    \arraybackslash%
  }X%
}
\newcolumntype{R}[1]{%
  >{%
    \hsize=\dimexpr#1\hsize\relax%
    \linewidth=\dimexpr\hsize\relax%
    \raggedleft\arraybackslash%
  }X%
}

% ------------------------------------------------------------------------------
% Paragraph notes.

\RequirePackage[restart,symbol]{parnotes}

% ------------------------------------------------------------------------------
% Enhancements for lists.

\RequirePackage[inline]{enumitem}

\newlist{Items}{itemize}{2}
\setlist[Items,1]{label={\textbullet}}
\setlist[Items,2]{label={\==}}
\setlist[Items]{%
  nosep,%
  leftmargin=*,%
}

\newlist{ItemsAfterDescriptionItem}{itemize}{2}
\setlist[ItemsAfterDescriptionItem,1]{label={\textbullet}}
\setlist[ItemsAfterDescriptionItem,2]{label={\==}}
\setlist[ItemsAfterDescriptionItem]{%
  nosep,%
  leftmargin=*,%
  before=\leavevmode\vspace*{-\dimexpr2\topsep+\baselineskip\relax},%
}

\newlist{UnlabeledItems}{itemize}{1}
\setlist[UnlabeledItems]{%
  label={},%
  labelsep=0pt,
  nosep,%
  leftmargin=*,%
}

% ------------------------------------------------------------------------------
% Load ‘siunitx’ for typesetting of units.

\RequirePackage{siunitx}
\sisetup{%
  detect-all,%
  detect-display-math,%
  binary-units,%
}

% \PrintDecimal[precision]{number} prints number rounded to precision.
% The default value of precision is 2.
% Integers will be converted to decimal; e.g., \PrintDecimal[2]{10} ⇒ 10.00.
\NewDocumentCommand{\PrintDecimal}{o m}{%
  \num[%
  round-mode=places,%
  round-integer-to-decimal,%
  round-precision=\IfNoValueTF{#1}{2}{#1},%
  ]{#2}%
}
% \PrintInteger{number} prints number as an integer.
% \PrintInteger{5}    ⇒ 5
% \PrintInteger{5.61} ⇒ 6
% \PrintInteger{5.47} ⇒ 5
\NewDocumentCommand{\PrintInteger}{m}{%
  \num[%
  round-mode=places,%
  round-precision=0,%
  zero-decimal-to-integer,%
  ]{#1}%
}

% ------------------------------------------------------------------------------
% Boxes.

\RequirePackage{adjustbox}
\RequirePackage{tcolorbox}
\tcbuselibrary{%
  most,%
  minted,%
}%

\NewDocumentCommand{\PrintInfo}{m m}{%
  \begin{adjustbox}{valign=m,minipage=.1\linewidth}
    \textnormal{#1}
  \end{adjustbox}%
  \hfill%
  \begin{adjustbox}{valign=m,minipage=.9\linewidth}
    #2
  \end{adjustbox}%
}
\NewDocumentEnvironment{UniformTCBRaster}{m}{%
  \begin{tcbraster}[%
  raster columns={#1},%
  raster halign=center,%
  raster valign=center,%
  raster every box/.style={%
    size=fbox,%
    halign=flush left,%
    valign=center,%
  },%
  ]%
}{%
  \end{tcbraster}%
}
\NewDocumentEnvironment{TCBUnlabeledItems}{}{%
  \begin{tcolorbox}\begin{UnlabeledItems}%
}{%
  \end{UnlabeledItems}\end{tcolorbox}%
}

% ------------------------------------------------------------------------------
% Hypertext and references.

\RequirePackage{hyperref}
\RequirePackage{cleveref}
\hypersetup{%
  unicode,%
  breaklinks,%
  colorlinks,%
  pdftitle={Dario Gjorgjevski’s curriculum vitæ},%
  pdfauthor={Dario Gjorgjevski},%
  pdfsubject={Curriculum vitæ},%
}

\NewDocumentCommand{\GmailURL}{m}{%
  \href{mailto:#1@gmail.com}{\nolinkurl{#1@gmail.com}}%
}
\NewDocumentCommand{\GitHubURL}{o m}{%
  \faGithub\href{https://github.com/\IfValueT{#1}{#1/}#2}{\nolinkurl{/#2}}%
}
\NewDocumentCommand{\SEFlair}{m m}{%
  \begin{adjustbox}{valign=m,minipage=\linewidth}
    \href{https://stackexchange.com/users/#1}{%
      \adjustimage{frame=1pt,width=\linewidth}{#2}%
    }
  \end{adjustbox}%
}

% ------------------------------------------------------------------------------
% Use BibLaTeX with Biber to process the bibliography.

\RequirePackage[%
style=numeric,%
backend=biber,%
defernumbers=true,%
datamodel=supervisor,%
]{biblatex}
\addbibresource{Theses_and_Publications.bib}
\AtBeginBibliography{\small}

\newbibmacro*{name:bold}[2]{%
  \def\do##1{%
    \iffieldequalstr{hash}{##1}{\NameFont\listbreak}{}%
  }%
  \dolistloop{\boldnames}%
}

\xpretobibmacro{name:family}{%
  \begingroup\usebibmacro{name:bold}{#1}{#2}%
}{}{}
\xapptobibmacro{name:family}{%
  \endgroup%
}{}{}

\xpretobibmacro{name:given-family}{%
  \begingroup\usebibmacro{name:bold}{#1}{#2}%
}{}{}
\xapptobibmacro{name:given-family}{%
  \endgroup%
}{}{}

\xpretobibmacro{name:family-given}{%
  \begingroup\usebibmacro{name:bold}{#1}{#2}%
}{}{}
\xapptobibmacro{name:family-given}{%
  \endgroup%
}{}{}

\xpretobibmacro{name:delim}{%
  \begingroup\normalfont%
}{}{}
\xapptobibmacro{name:delim}{%
  \endgroup%
}{}{}

\forcsvlist{\listadd\boldnames}{%
  % Where does this value come from?
  %   1. Open the ‘\jobname.bbl’ file
  %   2. Find the author whose name should be typeset in bold
  %   3. Copy the ‘hash’ field below
  {90e97b67720ed4a478d19395e2c396d7}%
}

\DeclareLanguageMapping{english}{supervisor-english}
\newbibmacro*{thesissupervisor}{%
  \ifnameundef{supervisor}{}{%
    \ifnumgreater{\value{supervisor}}{1}{%
      \bibstring{jointsupervision}%
    }{%
      \bibstring{supervision}%
    }%
    \addspace\printnames{supervisor}%
  }%
}
\xpatchbibdriver{thesis}{%
  \printfield{type}%
}{%
  \printfield{type}%
  \newunit%
  \usebibmacro{thesissupervisor}%
}{}{}

% ------------------------------------------------------------------------------
% ‘\EducationEntry’
%
% Mandatory parameters:
%   1. Degree
%   2. Institution
%   3. Date
%   4. Location
%   5. GPA
%   6. Grading scale
%
% Optional parameters:
%   7. Citation key for thesis
%   8. Citation keys for other publications
%   9. Remaining notes

\NewDocumentCommand{\EducationEntry}{m m m m m m o o o}{%
  \begin{tabularx}{\linewidth}{P{1.35} R{0.65}}
    \PrintDegree{#1}                        & Class of #3 \\
    \PrintInstitution{#2}                   & #4          \\
    \multicolumn{2}{M{2}{2}}{%
    \IfValueT{#7}{Thesis: \citetitle{#7}~\autocite{#7}.\par}%
    \IfValueT{#8}{Publications: \autocites{#8}.\par}%
    GPA of \PrintDecimal{#5}; scale #6.%
    \IfValueT{#9}{\par #9}%
    }
  \end{tabularx}
  \par%
}

% ------------------------------------------------------------------------------
% ‘\WorkEntry’
%
% Mandatory parameters:
%   1. Position
%   2. Institution
%   3. Date
%   4. Location
%   5. Description

\NewDocumentCommand{\WorkEntry}{m m m m m}{%
  \begin{tabularx}{\linewidth}{P{1.25} R{0.75}}
    \PrintPosition{#1}        & \PrintISODate{#3}    \\
    \PrintInstitution{#2}     & #4                   \\
    \multicolumn{2}{M{2}{2}}{#5}
  \end{tabularx}
  \par%
}

% ------------------------------------------------------------------------------

\begin{document}
\begin{minipage}{0.75\textwidth}
  {\Huge\PrintName{Dario Gjorgjevski}\par}%
  \medskip%
  {\Large\TextDS{Curriculum vitæ}\par}%
  \medskip%
  {%
    \renewcommand*{\arraystretch}{1.5}%
    \begin{tabularx}{\textwidth}{P{1.3} P{1.2} P{0.5}}
      \PrintInfo{\faAt}{\GmailURL{dario.gjorgjevski}}%
      &
        \PrintInfo{\faPhone}{\phonenumber[foreign]{01525 8666837}}%
      &
        \GitHubURL{d125q}%
      \\
      \PrintInfo{\faEnvelope}{Klara\-/Franke\-/Straße 21\\10557 Berlin}%
      &
        \PrintInfo{\faKey}{\inputminted{shell-session}{PublicKey.shell-session}}%
      &
        \SEFlair{3671527}{Pictures/SE-Flair.png}
    \end{tabularx}%
  }
\end{minipage}%
\hfill%
\begin{minipage}{.25\textwidth}
  \center%
  \begin{tikzpicture}[baseline=(Dario.center)]
    \node[circle,draw,inner sep=1.5cm,black,fill overzoom image*={clip}{Pictures/Profile}] (Dario) {};
  \end{tikzpicture}
\end{minipage}

\vspace*{-\baselineskip}%
\section{Experience}%
\label{sec:experience}

\begin{minipage}[t]{0.575\textwidth}
  \vspace*{0cm}%
  \WorkEntry%
  {Data Engineer}%
  {\href{https://corporate.zalando.com/en}{Zalando SE}}%
  {2019-09/}% chktex 8
  {Berlin, DE}%
  {%
    Development\parnote{Including \(24 \times 7\) responsibilities starting
      \PrintISODate{2020-08}} of ETL pipelines which every day process %chktex 8
    over\--- \PrintInteger{750} thousand \emph{attribute}, \PrintInteger{60}
    million \emph{price}, \PrintInteger{100} million \emph{stock}, and
    \PrintInteger{500} million \emph{availability}\--- update events into a
    coherent data landscape used to report on key business metrics and train
    machine learning models.%
    \leavevmode\vspace*{20pt}\newline\begingroup%
    \tcbset{%
      enhanced,%
      size=fbox,%
      on line,%
      box align=center,%
      fonttitle=\small,%
      fontupper=\small,%
      fontlower=\small,%
      halign title=center,%
      halign upper=center,%
      halign lower=center,%
    }%
    \begin{minipage}{.2\linewidth}
      \begin{tcolorbox}[title=Nakadi,remember as=Kafka]
        JSON events\tcblower{}Kafka
      \end{tcolorbox}
    \end{minipage}%
    \hfill%
    \begin{minipage}{.2\linewidth}
      \begin{tcolorbox}[title=AWS,remember as=Spark]
        Databricks\tcblower{}Spark
      \end{tcolorbox}%
    \end{minipage}%
    \hfill%
    \begin{minipage}{.2\linewidth}
      \begin{tcolorbox}[remember as=Oracle,colframe=framered,colback=backred]
        Oracle\parnote{Oracle is being deprecated and migrated away from}
      \end{tcolorbox}%
      \\%
      \begin{tcolorbox}[remember as=S3]
        Delta Lake\tcblower{}S3
      \end{tcolorbox}
    \end{minipage}%
    \hfill%
    \begin{minipage}{.15\linewidth}
      \begin{tcolorbox}[remember as=Exasol]
        Exasol
      \end{tcolorbox}%
      \\%
      \begin{tcolorbox}[remember as=Presto]
        Presto
      \end{tcolorbox}%
      \begin{tikzpicture}[overlay,remember picture,>=Stealth,shorten >=1pt,shorten <=1pt]
        \coordinate (Spark-SSE)  at ($(Spark.south)!0.5!(Spark.south east)$);
        \coordinate (Spark-NNE)  at ($(Spark.north)!0.5!(Spark.north east)$);
        \coordinate (S3-SSW)     at ($(S3.south)!0.5!(S3.south west)$);
        \coordinate (Exasol-NNW) at ($(Exasol.north)!0.5!(Exasol.north west)$);
        \path[->] (Kafka)     edge                                                             (Spark)
                  (Spark)     edge[draw=framered]                                              (Oracle)
                  (Oracle)    edge[draw=framered]                                              (Exasol)
                  (Spark)     edge                                                             (S3)
                  (S3)        edge[<->]                                                        (Presto)
                  (S3-SSW)    edge[out=-150,in=-30] node[midway,below,scale=0.75] {SNS \& SQS} (Spark-SSE)
                  (Spark-NNE) edge[out=30,in=150]                                              (Exasol-NNW);
        \node[surround={(Spark)}{Airflow}]                          (Scheduling) {};
        \node[surround={(Exasol)(Presto)}{Superset\\MicroStrategy}] (Reporting)  {};
      \end{tikzpicture}
    \end{minipage}
    \endgroup\vspace*{\baselineskip}\parnotes{}%
  }

  \WorkEntry%
  {Data Engineer}%
  {\href{https://www.so1.ai/en/}{SO1 GmbH}}%
  {2018-07/2019-09}% chktex 8
  {Berlin, DE}%
  {%
    \begin{Items}
    \item Responsible for a self\-/hosted Vertica on Azure:
      \begin{Items}
      \item Defined \emph{data vault} and \emph{dimensional} models of client
        data.
      \item Developed ETL processes to ingest data from Blob Storage, Kafka, and
        SFTP servers.
      \item Wrote and optimized queries to monitor KPIs and to compute features
        for machine learning models.
      \item Implemented user\-/defined extensions to evaluate machine learning
        models (e.g., LightGBM) directly inside Vertica.
      \end{Items}
    \item Translated business rules to \emph{minimum cost maximum flow}
      problems, yielding \SI{15}{\percent} greater value than previous greedy
      algorithms.
    \item Developed a microservice for top\=/\(k\) nearest neighbor queries in
      real time by utilizing \emph{locality\-/sensitive hashing} with
      \emph{MinHash} signatures.
    \item Conducted technical interviews for data engineers.
    \end{Items}%
  }

  \WorkEntry%
  {Data Scientist}%
  {\href{http://infiniteanalytics.com/}{Infinite Analytics, Inc.}}%
  {2017-11/2018-06}% chktex 8
  {Skopje, MK}%
  {%
    \begin{Items}
    \item Scraped clients’ websites using Scrapy.
    \item Developed a Spark application to compute and visualize
      \emph{actionable insights} using over \PrintInteger{2} billion facts about
      \PrintInteger{50} million customers.
    \end{Items}%
  }

  \WorkEntry%
  {Research Intern}%
  {\href{https://epfl.ch/}{EPFL}, \href{http://lca.epfl.ch/}{LCA2}}%
  {2017-07/09}% chktex 8
  {Lausanne, CH}%
  {%
    \begin{Items}
    \item Implemented software agents with \mintinline{python}{asyncio} and
      Mininet in
      \href{https://www.epfl.ch/labs/desl-pwrs/smartgrid/t-recs/}{T\=/RECS}.
    \item Modeled smart grid power traces at a timescale of \SI{20}{\ms} using
      approaches based on wavelets and long\-/range dependence.
    \item Increased the resolution of mean\-/aggregated measurements using deep
      learning for super\-/resolution.
    \end{Items}%
  }
\end{minipage}%
\hfill%
\begin{minipage}[t]{.4\textwidth}
  \vspace*{0cm}%
  \WorkEntry%
  {Research Intern}%
  {\href{https://epfl.ch/}{EPFL}, \href{https://lasec.epfl.ch/}{LASEC}}%
  {2016-06/09}% chktex 8
  {Lausanne, CH}%
  {%
    Studied and improved upon the complexity of state\-/of\-/the\-/art solving
    algorithms for the \emph{Learning With Errors} (LWE) problem.%
  }
  \section{Strengths}%
  \label{sec:strengths}

  \subsection{Theoretical Knowledge}%
  \label{sec:theoretical-knowledge}

  \begin{UniformTCBRaster}{2}
    \begin{TCBUnlabeledItems}
    \item Algorithms
    \item Data structures
    \end{TCBUnlabeledItems}
    \begin{TCBUnlabeledItems}
    \item Probability theory
    \item Statistics
    \end{TCBUnlabeledItems}
    \begin{TCBUnlabeledItems}
    \item Databases
    \item Distributed systems
    \item Big data
    \end{TCBUnlabeledItems}
    \begin{tcboxedraster}[raster columns=1]{blankest}
      \begin{tcolorbox}
        Linear algebra
      \end{tcolorbox}
      \begin{tcolorbox}
        Cryptography
      \end{tcolorbox}
    \end{tcboxedraster}
  \end{UniformTCBRaster}
  \begingroup\small%
  Additional academic exposure to \emph{compiler design} and \emph{programming
    language theory} along with various Lisps, Haskell, and Standard ML\@.\par%
  \endgroup

  \subsection{\texorpdfstring{Hands\=/On Experience}{Hands-On Experience}}%
  \label{sec:hands-on-experience}

  \begin{UniformTCBRaster}{2}
    \begin{tcboxedraster}[raster columns=1]{blankest}
      \begin{TCBUnlabeledItems}
      \item Python
      \item R incl.\ \TextSCP{Tidyverse}
      \item SQL \& PL/SQL
      \item Scala
      \end{TCBUnlabeledItems}
      \begin{tcolorbox}
        \begin{minipage}[t]{.5\linewidth}
          \begin{UnlabeledItems}
          \item Spark
          \item Kafka
          \item Airflow
          \end{UnlabeledItems}
        \end{minipage}%
        \hfill%
        \begin{minipage}[t]{.5\linewidth}
          \begin{UnlabeledItems}
          \item SciPy stack
          \item ELK stack
          \item Redis
          \end{UnlabeledItems}
        \end{minipage}
      \end{tcolorbox}
    \end{tcboxedraster}
    \begin{TCBUnlabeledItems}
    \item GNU/Linux
    \item AWS
      \begin{Items}[labelindent=8pt,labelsep=4pt,leftmargin=*]
      \item EC2
      \item S3
      \item Lambda
      \item SNS \& SQS
      \item CloudFormation
      \end{Items}
    \item Databricks
    \end{TCBUnlabeledItems}
    \begin{tcolorbox}
      \LaTeX{} incl.\ Ti\emph{k}Z/PGF
    \end{tcolorbox}
    \begin{tcolorbox}
      Computer algebra
    \end{tcolorbox}
  \end{UniformTCBRaster}

  \begingroup\small%
  Adept at working in \emph{agile teams} with Git for version control and
  various CI/CD workflows.\par%
  \endgroup

  \section{Honors}%
  \label{sec:honors}

  \begin{itemize}[leftmargin=*]
  \item Graduated \emph{summa cum laude} with a perfect GPA from the
    \PrintInstitution{Ss.\ Cyril and Methodius University}.
  \item Best student paper
    for~\autocite{Gjorgjevski:Combining_LWE-Solving_Algorithms}.
  \item Scholarships to attend the
    \href{http://summerschool-croatia.cs.ru.nl/2016/}{2016} and
    \href{http://summerschool-croatia.cs.ru.nl/2017/}{2017} editions of the
    Summer School on Real\-/World Crypto and Privacy held in Šibenik, Croatia.
  % \item Dean’s list\---awarded to the top \SI{2.5}{\percent} students of
  %   Computer Science \& Engineering\---at the \PrintInstitution{Ss.\ Cyril and
  %     Methodius University}.
  \end{itemize}
\end{minipage}

\begin{minipage}[t]{.575\textwidth}
  \section{Education}%
  \label{sec:education}
  \EducationEntry%
  {Computer Science \& Engineering}%
  {\href{http://ukim.edu.mk/}{Ss.\ Cyril and Methodius University}}%
  {2017}%
  {Skopje, MK}%
  {10}%
  {from 5 (E/F) to 10 (A)}%
  [Gjorgjevski:ECC_in_the_Rank_Metric]%
  [%
  Gjorgjevski:Combining_LWE-Solving_Algorithms,%
  Gjorgjevski_Gjorgjevikj:Using_Distributed_Representations,%
  ]%
  [%
  Earned \PrintInteger{240} ECTS credits.  As a senior, conducted computational
  exercises, homework assignments, and exams in:
  \begin{description}[leftmargin=*,font=\normalfont\itshape,widest=Linear Algebra]
  \item[Linear Algebra] Least squares, linear codes, and low\-/rank
    approximations in \textsc{SageMath} and Mathematica®.
  \item[Statistics] Data visualization, Monte Carlo methods, inference,
    hypothesis testing, and linear regression in R.
  \item[Databases] ER models, relational algebra, and ANSI SQL\@.
  \end{description}
  Presentations available at \GitHubURL[d125q]{Presentations}.%
  ]

  \begin{tcolorbox}[size=fbox,fontlower=\small]
    Over \PrintInteger{40} \emph{Massive Open Online Courses} on topics related
    to game theory, probabilistic graphical models, Bayesian statistics,
    combinatorics, automata and formal languages, mathematical optimization,
    etc.%
    \tcblower%
    Certifications available at
    \GitHubURL[d125q]{Personal/tree/master/Certifications}.
  \end{tcolorbox}

  \printbibliography[type=thesis,title=Theses]

  \printbibliography[nottype=thesis,title=Publications]
\end{minipage}%
\hfill%
\begin{minipage}[t]{.4\textwidth}
  \section{Projects}%
  \label{sec:projects}

  \begin{description}[style=nextline,leftmargin=!,font=\sffamily\bfseries]
  \item[C\=/like language\(\,\to\,\)PostScript transpiler] Transpiler
    implemented in Flex and GNU Bison to translate a C\=/like language for
    \emph{turtle graphics} to PostScript.
  \item[Trusted timestamping] Flask application for a simple file\-/sharing
    service which also provides \emph{trusted timestamps} as specified in
    RFC3161 and implemented in OpenSSL\@.
  \item[AS\=/level robustness of the Internet over time] Simulation of random
    and targeted attacks on the Internet topology.  Jupyter Notebook and source
    code available at \GitHubURL[d125q]{Internet_Robustness}.
  \item[Predicting readmission of diabetic patients]
    \begin{ItemsAfterDescriptionItem}
    \item Learning from imbalanced data using \mintinline{R}{mlr}.
    \item \emph{Fully reproducible reporting} using \mintinline{R}{knitr}.
    \end{ItemsAfterDescriptionItem}
    Report available at \GitHubURL[d125q]{Diabetic_Patients}.
  \item[Survey of the MinRank problem] \textsc{SageMath} implementations of:
    \begin{Items}
    \item Algorithms for solving MinRank; and a
    \item Zero\-/knowledge authentication protocol based on MinRank.
    \end{Items}
    Report available at \GitHubURL[d125q]{MinRank}.
  \item[Checksum verification on LPC1769] C program to verify MD5 checksums of
    payloads stored on an LPC1769 microcontroller.  Won prize for highest
    throughput.  Source code available at \GitHubURL[d125q]{DMA_Workshop}.
  \item[Substitution ciphers for Macedonian text]
    \begin{ItemsAfterDescriptionItem}
    \item Create substitution ciphers; and
    \item Break substitution ciphers using Markov chain Monte Carlo (MCMC)
      methods based on unigram and bigram frequencies.
    \end{ItemsAfterDescriptionItem}
    Mathematica® Notebook and corpus with Macedonian text suitable for frequency
    analysis available at \GitHubURL[d125q]{Macedonian_Substitution_Ciphers}.
  \end{description}
\end{minipage}
\end{document}

% ------------------------------------------------------------------------------

% Local Variables:
% fill-column: 80
% mode: latex
% TeX-command-extra-options: "-shell-escape"
% TeX-engine: luatex
% TeX-master: t
% End:
